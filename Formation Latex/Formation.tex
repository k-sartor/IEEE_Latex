\documentclass[a4paper,11pt]{article}
\usepackage[utf8]{inputenc}
\usepackage[frenchb]{babel}
\usepackage{amsmath, amssymb} %Permet notamment des caractères mathématiques plus évolué et d'utiliser le fait que les équations ne soient pas numérotées
\usepackage{hyperref}			% Pour permettre au lien d'être clicables

\usepackage{blindtext}

% Commandes nécessaires pour générer une page de garde "basique"
\title{Introduction à Latex}	%On constatera que sans le package inputenc, le "à" n'apparait pas
\author{Isabelle Mainz et Kevin Sartor}
\date{\today}						% La date est dans la langue liée à babel
\usepackage{graphicx}				%pour inclure des images
\usepackage{listings}				%Pour écrire du Code Source

%A partir d'ici le document commence
\begin{document}
\maketitle							%Permet de générer à l'aide des informations précédentes la page de garde basique
\tableofcontents					%Génère la table des matières
\newpage							%on impose de passer à une autre page

\section{Une première section numérotée avec les bases "basiques"}
On commence par signaler le fait que le symbole pourcent est ce qui est utilisé pour faire des commentaires (dans le code source latex).

Dans le préambule (le code avant le begin document), on utilise des packages (les libraires utiles) et d'autres informations génériques), éventuellement des fonctions.

Ce qui est obligatoire est entre accolades et ce qui est optionnel est entre crochets, en gros les options.

Par défaut Latex ne prend pas en compte les espaces multiples comme ici        . Par contre au besoin voici quelques commandes pour en introduire \url{http://www.xm1math.net/doculatex/espacements.html}

L'url est réalisée à l'aide du package hyperref. Pour réaliser un retour à la ligne dans la sortie, il faut une ligne vide dans le code source

Un paragraphe avec du texte pseudo aléatoire: \blindtext 		%commande avec le package du même nom pour remplir de texte pseudo-aléatoire.
\section{Une deuxième section numérotée}
\section*{Une section non numérotée et non reprise dans la table des matières} % pareil pour les autres niveau de référence
La seule différence est l'étoile dans le code source.
\subsection{Un exemple de sous section}
\subsubsection{Un exemple de sous sous section}

Attention toutefois que la hiérarchie est valable ici pour la classe article (dans d'autres classes, on pourrait en avoir d'autres).

\paragraph{Un paragraphe}

\subparagraph{Un "sous" paragraphe}
On notera que par défaut les (sous-)paragraphes ne sont pas référencés dans la table des matières.

\blindtext

\newpage
\section{Les équations}

Il existe plusieurs possibilités pour écrire des équations. La première c'est d'écrire une équation directement dans le texte $I = \frac{U}{R}$ (loi d'Ohm). Cette équation peut être suivie de texte mais ne peut être liée à une référence (voir par la suite)

Une autre méthode est proposée ci-dessous (dans ce cas elle est centrée et non numérotée).
$$a_i^e = \frac{A}{B}$$

Une alternative numérotée et référençable
\begin{equation}
a_i^e = \frac{A}{B}
\end{equation}

Exemples avec des indices et exposants dont la longueur est supérieure à un caractère et une introduction au lettres grecques
\begin{equation}
a_{indice}^{exposant} = \frac{\omega}{\tau}
\end{equation}

Une alternative uniquement référençable (nécessite le package amsmath).
\begin{equation*}
a_i^e = \frac{A}{B}
\end{equation*}

\newpage
\section{Les listes}
\subsection{La liste simple}
Premier type de listes (avec tirets)
\begin{itemize}
	\item Mon premier élément
	\item Mon deuxième élément
	\begin{itemize}
		\item [$\circ$] Un nouvel élément avec un autre type de "bullet"
		\item [$\bullet$] Encore un autre...
	\end{itemize}
\end{itemize}
\subsection{Les listes énumérées}
Une liste avec des nombres
\begin{enumerate}
	\item Mon premier élément
	\item Mon second élément
\end{enumerate}
Et comme pour d'autres comportements, on peut redéfinir une commande (ici la numérotation d'une liste énumérée)
\renewcommand{\theenumi}{\Roman{enumi}}

\begin{enumerate}
	\item Mon premier élément
	\item Mon second élément
\end{enumerate}

\newpage
\section{Les références\label{sec:refs}}
	
	\subsection{Référence à une équation}
	Voici une équation en ligne $a = b$ et une en bloc \[a = b.\] Pour pouvoir y faire référence il faut utiliser
	
	\begin{equation}
		a = b. \label{eqn:mon-equation}
	\end{equation}
	
	Et je peux maintenant y faire référence comme ça \ref{eqn:mon-equation} ou comme ça \eqref{eqn:mon-equation}.
	
	\subsection{Référence à une section\label{sec:ss-ref}}
		
	Voici une référence à ma section: \ref{sec:refs}.
	Ou bien à ma sous-section: \ref{sec:ss-ref}
	
	\blindtext
	
	Voici une note en bas de page\footnote{ma note}.
	Voici une deuxième\footnote{2e note}.
	
	\newpage
	
	\section{Les images}
	
	Image qui a la largeur de la page
	
	\includegraphics[width=\textwidth]{images/nyan-cat.png}
	
	\vspace{1cm}
	
	On impose la largeur en cm
	\includegraphics[width=3cm]{images/nyan-cat.png}
	
	\includegraphics[height=2in]{images/nyan-cat.png}
	
	\includegraphics[height=2in, width=3cm]{images/nyan-cat.png}

	L'option scale pour définir un ratio à l'image (les proportions sont conservées).
	
	\includegraphics[scale=0.15]{images/nyan-cat.png}
	
	\blindtext
	
	\newpage
	\begin{figure}[hbt]%h b t p
		\centering
		\includegraphics[scale=0.15]{images/nyan-cat.png}
		\caption{Une image}
		\label{img:mon-image}
	\end{figure}
	
	Je peux faire référence à mon image: \ref{img:mon-image}.
	
	\newpage
	\section{Les tableaux}
	
	Voici un premier tableau:
	
	\begin{tabular}{|l|| c |r|} %l r c
		\hline
		Nom & Contenance & Stock\\
		\hline
		\hline
		Blanche & 33cl & 25\\
		Jupiler & 33cl & 149\\
		Guinness & 1pint & 5\\
		\hline
	\end{tabular}
	
	\vspace{1cm}
	
	
	\begin{table}[h]
		\centering
	\begin{tabular}{|p{3cm}|c|r|r|}
		\hline
		Nom & Contenance & Quantité & Stock\\
		\hline
		\hline
		Blanche & 33cl & 30 & 25\\
		\cline{2-2}
		Jupiler & 33cl & \multicolumn{2}{c|}{à volonté}\\
		\cline{3-4}
		Guinness & 1 pint & 5 & 5\\
		\hline
	\end{tabular}
	\caption{Tableau des bières}
	\label{tab:bieres}
	\end{table}
	
	Et la référence \ref{tab:bieres}
	
	\newpage
	\section{Writing Source Code in \LaTeX}
	
	\lstinputlisting[language=C]{QuickSort.c}
	
	Plus d'informations: \url{https://en.wikibooks.org/wiki/LaTeX/Source_Code_Listings} (entre autre pour le Syntax Highlighting)
	
	Une alternative: \url{https://en.wikibooks.org/wiki/LaTeX/Algorithms}

\newpage
\section{Quelques liens utiles}
De l'aide en général sur ces liens:
\url{http://www.grappa.univ-lille3.fr/FAQ-LaTeX/} et \url{https://fr.wikibooks.org/wiki/LaTeX}

De l'aide pour connaitre le code de symbole
\url{http://detexify.kirelabs.org/classify.html}

En règle générale, il est possible d'interroger son moteur de recherche préféré avec une question du type: "Latex note de bas de page en haut" et de trouver très rapidement une solution à son problème. Comme souvent, une question en anglais aura un plus grand nombre de résultats

Pour terminer un site vous permettant d'écrire du latex (en accès ouvert) dans un explorateur web \url{https://fr.wikibooks.org/wiki/LaTeX}
\end{document}

